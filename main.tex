\documentclass[a4paper, oneside]{book}

%% Language and font encodings
\usepackage[frenchb]{babel}

\usepackage[utf8]{inputenc}
\usepackage[T1]{fontenc}

\usepackage{verbatim}
\usepackage{enumitem}
\usepackage[colorlinks=true, allcolors=blue]{hyperref}

% creation de la commande circled, mot entoure d'un cercle
\newcommand*\circled[1]{\tikz[baseline=(char.base)]{% <---- BEWARE
            \node[shape=circle,draw,inner sep=2pt, color=pumpkin] (char) {#1};}}

% Using \hfuzz allows \hbox to be overfull to the given amount before a warning is raised
\hfuzz = 5pt

% quotes package
\usepackage[autostyle, maxlevel = 2]{csquotes}

% bibliography package            
\usepackage[backend = biber, style = numeric]{biblatex}   
\addbibresource{references.bib}            

% glossary package
\usepackage{nameref}
\usepackage[toc, numberedsection=nameref]{glossaries}

% loading glossary file
\loadglsentries{Glossaire/glossaire.tex} 
\makeglossaries

% package for references and links 
\usepackage{caption}

% package for insertion of pdf pages
\usepackage{pdfpages}

% package to change behavior of floats numbering
\usepackage{chngcntr}
\counterwithout{figure}{chapter}

% pour les images
\usepackage{graphicx}
\usepackage{caption} 
\usepackage{float} 
\usepackage{wrapfig}

%% pour les tableaux
\usepackage{array}
\usepackage{tabularx}
\usepackage{multirow}
\usepackage{slashbox}
\usepackage{colortbl}
\usepackage{framed}
\usepackage{adjustbox}

% pour les maths
\usepackage{amsmath}
\usepackage{amsfonts}
\usepackage{amssymb}
\usepackage{mathrsfs}  
\usepackage{pifont}
\newcommand{\cmark}{\ding{51}}
\newcommand{\xmark}{\ding{55}}

%% package for landscape page view
\usepackage{pdflscape}
\usepackage{rotating}

%% pour dessiner des graphiques et des schemas
\usepackage{tikz}
\usepackage{tkz-graph}
\usetikzlibrary{calc, decorations.pathreplacing}

% Pgfplots
\usepackage{pgfplotstable}
\usepackage{pgfplots}
\usetikzlibrary{pgfplots.groupplots}
\pgfplotsset{compat=1.12}

% définitions des couleurs
\usepackage{color}
  \definecolor{grey}{rgb}{0.4,0.4,0.4}
  \definecolor{blue}{rgb}{0.2,0.3,0.6}
  \definecolor{teal}{rgb}{0.1,0.4,0.4}
  \definecolor{green}{rgb}{0.1,0.7,0.2}
  \definecolor{red}{rgb}{0.8,0.1,0.2}
  \definecolor{pumpkin}{rgb}{0.9, 0.3, 0}

%% Pour l'intégration de code SQL
\usepackage{listings} 
\usepackage{listingsutf8}
\lstloadlanguages{JAVA, SQL}
\lstset{ % affichage du code par défaut
    inputencoding=utf8/latin1,
    basicstyle=\footnotesize\sf,
    morecomment=[s]{/*}{*/},
    morecomment=[l]{//}, 
    keywordstyle=\sffamily\bfseries\color{teal},
    commentstyle=\itshape\color{grey},
    stringstyle=\rmfamily\color{pumpkin},
    tabsize=2, frame=single, breaklines=true,
    showspaces=false, showstringspaces=false,extendedchars=true, 
    numbers=left, numberstyle=\tiny,
    extendedchars=true,
    literate={\$}{{{\$}}}1 {é}{{\'e}}1,    
}

\AtBeginDocument{\counterwithout{lstlisting}{chapter}}

%% definition de style pour une ligne entiere d'un tableau
\newcolumntype{+}{>{\global\let\currentrowstyle\relax}}
\newcolumntype{^}{>{\currentrowstyle}}
\newcommand{\rowstyle}[1]{\gdef\currentrowstyle{#1}%
#1\ignorespaces
}

%% Sets page size and margins
\usepackage[a4paper,top=2cm,bottom=2cm,left=2cm,right=2cm,marginparwidth=1.75cm]{geometry}

%% Pour ecrire des algorithmes
\usepackage[vlined,ruled,linesnumbered]{algorithm2e}

%% Useful packages
\usepackage{amsmath}
\usepackage{amssymb}
\usepackage{amsthm}
\theoremstyle{definition}
\newtheorem{example}{Example}

\usepackage[colorinlistoftodos]{todonotes}

\usepackage{titlesec}
\titleformat{\chapter}[display]{\flushleft\Huge\itshape}{\quad}{0.5em}{}[]
\titleformat{\paragraph}[runin]{\normalfont\normalsize\bfseries}{}{0pt}{}

% leaves out chapter numbers in section numbering
\renewcommand*\thesection{\arabic{section}}

% command defining new chapter type, to be used whith roman page numbering
\newcommand\romanchapter[1]{
  \chapter*{#1}
  \markboth{\MakeUppercase{#1}}{}
  \addcontentsline{toc}{chapter}{#1}
}

\usepackage{fancyhdr}
\setlength{\headheight}{15.2pt}
\pagestyle{fancy}
\rhead{} % empty right header

\begin{document}

\begin{titlepage}
\begin{center}
\begin{sffamily}

{\large
Faculté de Sciences de Montpellier \\[.5cm]
Master 2 AIGLE\\2017 -- 2018\\[2cm]
}


% Title
\rule{\textwidth}{1.6pt}\vspace*{-\baselineskip}\vspace*{2pt} 
\rule{\textwidth}{0.4pt}\\[\baselineskip]
{\LARGE
Notation symbolique de flux de contrôle musicaux et multimédias\\[0.7\baselineskip]
\includegraphics[width=0.3\textwidth]{Paratextes/i/logo.png}
\\[0.5\baselineskip]
Étude bibiliographique
}\\[0.2\baselineskip] 
\rule{\textwidth}{0.4pt}\vspace*{-\baselineskip}\vspace{3.2pt}
\rule{\textwidth}{1.6pt}\\[\baselineskip]
\vspace*{2\baselineskip}

% Author and supervisor
\noindent
\begin{center}
     \large
    \emph{\textbf{Étudiant:}}\\
    Vincent Iampietro \\
    \smallskip
    \large
    \emph{\textbf{Encadrant:}}\\
    Jean Bresson\\
    \emph{\textbf{Co-encadrant:}}\\
    Rama Gottfried
\end{center}%



\end{sffamily}
\end{center}
\end{titlepage}

%%%%% PAGE DE GARDE %%%%%
\pagenumbering{Roman}

\setcounter{tocdepth}{1} % profondeur du sommaire, 1 pour sections uniquement

%%%%% SOMMAIRE %%%%%%
\tableofcontents
\clearpage

%%%%% INTRODUCTION %%%%%
\chapter{Introduction}
La notation musicale peut être distinguée en deux approches : une approche prescriptive et une approche descriptive \cite{battier2015}.
La notation prescriptive a pour but de décrire \og comment la musique doit sonner \fg.
Dans cette optique, la partition fait office de référence ou du moins de repère pour l'interprétation d'une pièce. 
La notation descriptive tente de retranscrire \og comment la musique a sonné \fg.
Ainsi, à des fins d'analyse, une pièce peut être caractérisée et fixée sur la \gls{portee}.
Également, la partition est l'outil privilégié du compositeur pour la communication de sa musique, et, au-delà de l'objet fini, constitue un espace de travail.
Pour donner un exemple, la préparation d'une pièce contemporaine par un ensemble musical se fait souvent en collaboration avec le compositeur. La partition constitue alors le support de la discussion et se voit même être modifiée pour les besoins de l'exécution de la pièce. Comme le dit Carmine E. Cella, chercheur et compositeur à l'IRCAM, en parlant de l'écriture musicale : \og le créateur doit s'efforcer de trouver un compromis notationnel entre sa pensée et la réalisation pratique de sa pièce \fg. L'annexe \ref{sec:refletsDeLOmbre} donne deux exemples de partitions représentant la même partie de la pièce \textit{Reflets de l'ombre} (C. E. Cella, 2013), montrant l'adaptation de la notation à des fins d'exécution.

Aussi, procurer des outils aux compositeurs pour leur permettre de noter leurs œuvres est un enjeu primordial. Bien sûr, ces outils ont besoin d'être en phase avec les pratiques de création de leur époque, pour ne pas brider la créativité des utilisateurs.
Or, la musique contemporaine savante et la composition multimédia soulèvent de nouveaux défis d'écriture, en apportant des pratiques inédites, fortement liées à l'usage massif de l'informatique et de l'électronique par les compositeurs. Par conséquent, la présente étude dresse un état de l'art des outils informatiques pour la notation musicale en se posant la question de la pertinence de ces outils pour la transcription des pièces contemporaines.

D'abord, la notation musicale sera décrite sous un axe historique, en s'attardant sur la période  contemporaine et les problématiques notationnelles qu'elle soulève. Ensuite, une vue d'ensemble des moyens informatiques pour la création musicale sera présentée, afin de donner un aperçu au lecteur de l'écosystème actuel dans lequel les compositeurs évoluent et qui influence de fait leurs pratiques. Puis, les outils informatiques existants pour une notation symbolique et traditionnelle de la musique seront détaillés, et leurs caractéristiques seront discutées sous l'angle de la problématique. Enfin, la dernière section s'intéressera aux logiciels et environnements abordant la notation musicale sous un axe contemporain, apportant une dimension interactive aux partitions.

Pour le confort de lecture et la compréhension du rapport, les mots du vocabulaire musical apparaissent en bleu et sont répertoriés dans le glossaire à la fin du document (voir \nameref{main}).




\stepcounter{page}
\pagenumbering{arabic}

%%%%% CHAPTER "PRESENTATION DU PROJET SYMBOLIST" %%%%%
\chapter{Présentation du projet symbolist}
\label{chap:presentationSymbolist}
\todo[inline]{introduction pour le chapitre}
	
\section{Genèse et objectifs du projet symbolist}
\label{sec:geneseSymbolist}
En qualité de compositeur, Rama Gottfried a été confronté à des problèmes de notation qui ont conduit à l'élaboration du projet \textit{symbolist} de paire avec Jean Bresson.
Rama Gottfried a longtemps utilisé le logiciel \textit{Sibelius}\footnote{\url{https://www.avid.com/fr/sibelius}} pour écrire ses pièces musicales et multimédias. Cependant, la trop grande limitation du logiciel quant à l'écriture d'œuvres contemporaines lui a fait abandonner son utilisation. 
Rama Gottfried opère alors un retour à l'usage du papier pour composer, le support offrant une liberté bien plus importante. Cependant, la numérisation des partitions \og papier \fg engendre une quantité relativement importante de données, représentant un inconvénient certain pour le partage des documents.
En conséquence, Rama Gottfried se tourne vers le logiciel \textit{Adobe Illustrator} pour ses qualités d'application de dessin vectoriel\footnote{\url{https://www.adobe.com/fr/products/illustrator.html}}. \textit{Illustrator} permet au compositeur d'entretenir la relation qu'il avait avec le support papier. L'interface de l'application s'abstrait du cadre de la portée et permet un mode d'expression graphique plus direct. De plus, le plugin \textit{scriptographer} d'\textit{Illustrator} permet à R.Gottfried de créer ses propres outils de dessin \cite{scriptographer2018}. Plus spécifiquement, \textit{scriptographer} permet d'agrémenter la palette d'outils \textit{Illustrator}; un outil est défini par un ensemble de scripts Javascript, qui seront lancés en réponse à des évènements \og souris \fg. Les scripts génèrent des éléments graphiques dans le canevas \textit{Illustrator}. Malheureusement, le plugin \textit{scriptographer} pour \textit{Illustrator} n'est aujourd'hui plus supporté par son équipe de développement. C'est à ce moment là que Rama Gottfried et Jean Bresson ont initié le projet \textit{symbolist}.

\paragraph{Objectifs du projet} \textit{symbolist} propose un environnement pour la notation graphique de pièces musicales et multimédias \cite{gottfried2018}. Bien sûr, les logiciels \textit{wysiwyg} comme \textit{Sibelius} ou \textit{Finale} proposent déjà de noter graphiquement la musique, en utilisant la notation traditionnelle occidentale. C'est sur ce point que ces programmes diffèrent.
Le premier objectif de \textit{symbolist} est de se rapprocher de la composition sur feuille blanche, aussi le compositeur ne se voit imposer aucun système de notation. L'utilisation de la \gls{portee} n'est, par exemple, plus obligatoire.
L'impossibilité de standardiser la notation des œuvres contemporaines a fait privilégier aux créateurs de \textit{symbolist} l'aspect de libre création graphique plutôt que d'imposition d'un canevas notationnel.
Aussi, \textit{symbolist} propose des outils de dessins standards (lignes, formes géométriques...) pour la création de symboles, et permet à l'utilisateur de stocker les symboles créés dans une palette pour réutilisation.
Même si l'interface graphique se détache de la portée, l'idée d'ordonnancer temporellement les symboles créés est conservée. En effet, dans \textit{symbolist}, chaque symbole peut être associé à un symbole de type \textit{staff} (une voix de la portée, en anglais) qui fait office de référence temporelle. Ainsi, les symboles associés à un \textit{staff} se voient attribuer un marqueur de temps selon leur position sur l'axe horizontal. Dans une partition \textit{symbolist}, il est admis que l'axe horizontal représente le temps, même si un symbole n'est marqué temporellement qu'après avoir été associé à un \textit{staff}. En revanche, et contrairement à la portée, l'axe vertical ne représente pas systématiquement la hauteur des sons. Un symbole placé au-dessus d'un autre ne doit pas être interprété comme produisant un son plus aigu. La sémantique de l'axe vertical est laissé à la discrétion du compositeur.

Un autre objectif de \textit{symbolist} est de tirer des symboles graphiques une information interprétable par la machine, qui permettrait de contrôler les paramètres d'une pièce musicale ou multimédia. Inversement, \textit{symbolist} peut être vu comme un transcripteur symbolique des variations des paramètres d'une pièce, recevant de l'information depuis des processus extérieurs et transformant cette information en symboles dans la partition.   
Pour ce faire, et en s'inspirant du format SVG qui décrit des formes graphiques en syntaxe XML \cite{svg2011}, chaque symbole d'une partition \textit{symbolist} est défini par un bundle OSC \cite{wright2002}. Comme présenté dans l'étude bibliographique, un bundle OSC est un ensemble de messages OSC, couples clé-valeur où la clé est exprimée sous forme d'url.
Le listing \ref{lst:bundleOSCMin} présente le bundle OSC minimal décrivant les informations partagées par l'ensemble des symboles d'une partition.

\begin{lstlisting}[language=java, 
				   caption={Messages OSC \'{e}l\'{e}mentaires pour les symboles d'une partition \textit{symbolist}}, 
				   label={lst:bundleOSCMin}, 
				   captionpos={b}, 
				   numbers=none]
/type  "circle"
/name  "myCircle"
/id    "circle/1"
/staff "staff/0"
/w     30
/h     30
/x     150
/y     160 
\end{lstlisting}

Le bundle OSC présenté dans le listing \ref{lst:bundleOSCMin} décrit un cercle nommé \textit{myCircle}, dont l'identifiant est \textit{circle/1}. Le cercle est attaché à un \textit{staff} (référent temporel) dont l'identifiant est \textit{staff/0}. Graphiquement, la figure est définie dans une boîte de largeur 30 pixels et de hauteur 30 pixels. Le centre de la boîte est situé aux coordonnées $(150, 160)$.
L'avantage de décrire les symboles sous forme de bundle OSC réside dans le caractère standard du protocole, permettant de fait l'envoi/réception de données structurées à/depuis un autre programme. Les possibilités d'interfaçage du logiciel \textit{symbolist} s'en voient décuplées.
Dans une optique d'interaction avec d'autres programmes, l'application a pour finalité d'être déployer en tant qu'objet \textit{Max} \cite{puckette1991} et \textit{OpenMusic} \cite{agon1998}, deux langages de programmation visuelle pour l'informatique musicale.	

\section{Analyse de l'existant}
\label{sec:analyseSymbolist}
Au début de ce stage, le logiciel \textit{symbolist} en est déjà à un état de développement avancé. Cette section détaille l'état de l'application d'un point de fonctionnel, technologique et architectural, tel qu'il se trouvait au début de la période de formation.

\subsection{Interface et fonctionnalités existantes}
\label{subsec:uiAndExistingFeatures}

Au commencement du stage, l'interface graphique de \textit{symbolist} offre déjà les fonctionnalités basiques de dessin sur l'espace de la partition. Les différentes vues composants l'éditeur graphique de \textit{symbolist} sont présentées en figure \ref{fig:symbolistUIBefore}.

\begin{figure}[H]
	\centering
	\includegraphics[keepaspectratio=true, width=\textwidth]{PresentationDeSymbolist/i/symbolistUIBefore.png}
	\caption{Éditeur graphique de \textit{symbolist}}
	\label{fig:symbolistUIBefore}
	\small
	\it
	En \textcolor{green}{vert}, la palette des symboles disponibles et dessinables sur la partition. En \textcolor{red}{rouge}, la partition, ou l'espace de composition et d'édition des symboles. En \textcolor{blue}{bleu}, l'inspecteur de symboles, qui présente le bundle OSC associé au symbole graphique couramment sélectionné (symbole surligné en bleu dans la partition).
\end{figure}

L'éditeur graphique de \textit{symbolist} permet à l'utilisateur de dessiner directement sur la partition à l'aide de la souris, dans une modèle d'interaction \textit{wysiwyg} (\textit{what you see is what you get}). Dans \textit{symbolist}, la partition fait référence à l'espace graphique où les symboles sont placés et édités par l'utilisateur. De manière sous-jacente, la partition est stockée comme une liste de symboles. Les symboles de la partition qui sont attachés à un référent temporel, un \textit{staff}, se voient attribuer une valeur temporelle de départ et de fin correspondant à leur point de départ et de fin sur l'axe horizontal. Ces symboles sont ordonnancés temporellement dans la liste reflétant la partition.

De plus, l'éditeur graphique n'est pas le seul moyen d'interagir avec le logiciel \textit{symbolist}. En effet, \textit{symbolist} est également déployer dans une version \textit{objet} pour l'environnement \textit{Max} et l'environnement \textit{OpenMusic}. Dans ces environnements, l'application \textit{symbolist} est représentée comme une boîte, recevant et envoyant des messages à d'autres objets. Les différents messages auxquels répond l'objet \textit{symbolist} dans l'environnement Max sont présentés en figure \ref{fig:symbolistMaxObject}.

\begin{figure}[H]
	\centering
	\includegraphics[keepaspectratio=true, width=0.8\textwidth]{PresentationDeSymbolist/i/symbolistMaxObject.png}
	\caption{L'objet \textit{symbolist} dans Max}
	\label{fig:symbolistMaxObject}
	\small
	\it
	En \textcolor{red}{rouge}, l'objet \emph{symbolist} représenté, comme tous les objets \emph{Max}, sous forme de boîte. En \textcolor{green}{vert}, les messages pouvant être envoyés à l'objet \emph{symbolist}.
	En \textcolor{blue}{bleu}, un afficheur de bundle OSC, proposé par une librairie indépendante, témoin des messages de sortie générés par l'objet \emph{symbolist}.  
\end{figure}

Les messages\footnote{Dans Max, un message peut être envoyé à un objet via l'objet \textit{message}, qui n'est autre qu'un bouton cliquable avec un label, à connecter à l'entrée d'un receveur.} compris par l'objet \textit{symbolist} permettent de lire et d'écrire des symboles dans la partition.
Comme exemples de messages de lecture, \lstinline|getsymbol n|, lit le \textit{n-ième} symbole de la partition, \lstinline|time t|, lit le contenu de la partition au temps \textit{t}…
Le résultat de la lecture est envoyé sur la sortie de l'objet \textit{symbolist}.

L'écriture de symboles dans la partition se fait par l'envoi de bundles OSC en entrée de l'objet \textit{symbolist} (voir la figure \ref{fig:symbolistMaxObject}, en haut à droite). 
Enfin, l'éditeur graphique peut être lancé par l'envoi du message \lstinline|open|.

\textit{symbolist} est également déployé dans une version objet \textit{OpenMusic}. La figure \ref{fig:symbolistOMObject} montre un exemple de computation de partition avec l'objet \textit{symbolist} (nommé \textit{sym-score}) dans \textit{OpenMusic}.

\begin{figure}[H]
	\centering
	\includegraphics[keepaspectratio=true, width=\textwidth]{PresentationDeSymbolist/i/symbolistOMObject.png}
	\caption{L'objet \textit{symbolist} dans OpenMusic}
	\label{fig:symbolistOMObject}
	\small
	\it
	En \textcolor{red}{rouge}, l'objet \emph{symbolist} \og sym-score \fg et l'éditeur graphique associé.
	En \textcolor{green}{vert}, les objets \emph{OpenMusic} servant à la création de bundles OSC envoyés à l'objet \emph{sym-score}.
\end{figure}

L'objet \textit{symbolist} dans sa version \textit{OpenMusic} reçoit des bundles OSC et crée les symboles correspondant dans la partition. Ensuite, le contenu de la partition peut être envoyé sur la sortie, en utilisant la barre de lecture \textit{OpenMusic} (activée par la touche espace).    

\paragraph{Fonctionnalités existantes} Afin de dresser un bilan des fonctionnalités implantées dans \textit{symbolist}, des \textit{user-stories}\footnote{Une \textit{user-story} est une manière d'exprimer une fonctionnalité logiciel, répandue chez les praticiens de la philosophie Agile. Une \textit{user story} prend la forme: \og En tant que tel type d'utilisateur, j'effectue telle action dans tel but \fg. De cette manière, une fonctionnalité est exprimée du point de vue de l'utilisateur, évitant l'écueil d'une formalisation trop technique. } ont été écrites à partir des possibilités du logiciel.
La liste des \textit{user stories} implantées dans \textit{symbolist} au début du stage est la suivante:
\begin{itemize}[label=--]
	\item En tant que compositeur, je dessine des courbes pour créer de nouveaux symboles.
	\item En tant que compositeur, je dessine des formes géométriques pour créer de nouveaux symboles.
	\item En tant que compositeur, j'écris du texte pour créer de nouveaux symboles ou pour annoter ma partition.
	\item En tant que compositeur, j'édite les propriétés d'un symbole existant pour définir sa forme.
	\item En tant que compositeur, je groupe des symboles entre eux pour créer un nouveau symbole.
	\item En tant que compositeur, je transforme un symbole en \textit{staff}, pour définir une référence temporelle dans la partition.
	\item En tant que compositeur, j'associe un symbole à un \textit{staff} pour lui procurer une valeur temporelle.
	\item En tant que compositeur, je peux ajouter un de mes propres symboles à la palette pour le réutiliser ensuite.
	\item En tant que compositeur, je peux annuler/recommencer les actions effectuées sur la partition pour la maintenir dans une état cohérent.
	\item En tant que compositeur, je peux lire le contenu de ma partition à un temps $t$. 
\end{itemize}

Les \textit{user stories} concernant le dessin sur la partition ont été réparties en catégories. En effet, chaque type de symboles est accompagné de problématiques spécifiques, ce qui fait considérer la création de texte, de courbes ou de formes prédéfinies comme des fonctionnalités à part entière.

\subsection{Framework de développement pour l'application symbolist}
\label{subsec:frameworkAndTechnologies}
\textit{symbolist} est développé avec le framework \textit{C++} \textit{JUCE} \cite{juce2018}. L'environnement \textit{JUCE} est un standard pour le développement d'applications liées à l'audio. Dans le cas de \textit{symbolist}, \textit{JUCE} a été utilisé pour ses capacités de gestion des bundles OSC et création d'interface graphique. La librairie \textit{odot} a ensuite été préférée pour la gestion des bundles OSC dans \textit{symbolist}. En effet, \textit{odot} augmente les capacités du format OSC, en lui adjoignant par exemple un léger langage de programmation. Aujourd'hui \textit{JUCE} n'est donc plus utilisé dans \textit{symbolist} que pour ses composants graphiques. 

En plus de son framework de développement, l'application \textit{symbolist} est également destinée à être déployée dans les deux environnements que sont \textit{Max} et \textit{OpenMusic}. Aussi, la construction de l'application \textit{symbolist} en tant qu'objet \textit{Max} et \textit{OpenMusic} est détaillée ci-après.

\paragraph{Création d'un objet Max} Max propose une API pour la création de nouveaux objets \cite{maxApi2018}. L'API Max est écrite en \textit{C}, ainsi que doivent l'être les objets créés par les utilisateurs pour étendre le système.
Ces objets sont appelés des \textit{externals} dans Max.
Ils doivent posséder une certaine structure pour pouvoir être compilé et utilisé par la suite:
\begin{itemize}[label=--]
	\item Le fichier de définition de l'external doit inclure le header \textit{ext.h} à la première ligne.
	\item Le nouvel objet défini doit être déclaré comme une structure \textit{C} avec un premier champ de type spécial.
	\item Une méthode de création et de libération de l'objet doivent être définies, ainsi qu'une méthode pour chaque message que peut recevoir l'objet.
	\item Ces méthodes sont liées à l'objet au sein de la procédure \lstinline|ext_main| qui fait office de procédure d'initialisation. De fait, \lstinline|ext_main| est appelée lorsque l'objet cible est appelé dans un patch Max, c'est à dire lorsque son nom est tapé dans une boîte.
\end{itemize}

L'application \textit{symbolist} étant implémentée en \textit{C++}, une API équivalente en \textit{C} a été écrite afin de pouvoir communiquer avec \textit{Max} et \textit{OpenMusic}.
Les méthodes fournies par l'API \textit{C} \textit{symbolist} sont présentées en annexe , page .
\todo[inline]{Ajouter l'API C symbolist en annexe}  

\paragraph{Création d'un objet OpenMusic} \textit{OpenMusic} pouvant être vu comme une surcouche graphique du langage \textit{Common Lisp} et de son système objet \textit{CLOS} \cite{bresson2009}, la création de nouveaux objets \textit{OpenMusic} est équivalent à la définition d'une nouvelle classe \textit{Lisp}.
Par exemple, l'objet \textit{symbolist} dans \textit{OpenMusic} est définie comme une classe possédant deux attributs: l'attribut \textit{symbols}, une liste de bundles OSC représentant la partition, et l'attribut \textit{palette-symbols}, une autre liste de bundles OSC représentant la palette des symboles disponibles.  
De fait, la boîte représentant l'objet \textit{symbolist} dans \textit{OpenMusic} possède trois points d'entrée et de sortie. Le premier point d'entrée correspond à la définition par copie d'un objet \textit{symbolist} par un autre objet \textit{symbolist}; le point d'entrée est dénommé \textit{self}. Les deux autres points d'entrées correspondent aux attributs de l'objet \textit{symbolist}, \textit{symbols} et \textit{palette-symbols}.
Les mêmes points se retrouvent symétriquement définis en sortie.

Un objet \textit{OpenMusic} peut également être muni d'un éditeur, c'est à dire une fenêtre graphique apparaissant lors d'un double-clic sur l'objet. L'éditeur associé à un objet \textit{OpenMusic} est défini par une classe \textit{Lisp} étendant la classe \textit{OMEditor}. 
Dans le cas de l'objet \textit{symbolist}, son éditeur fait la liaison avec l'API \textit{C} permettant de contrôler avec l'application \textit{symbolist} et son interface.
 
\subsection{Architecture de l'application symbolist}
\label{subsec:architectureBefore}
Le framework \textit{JUCE}, utilisé pour développer l'application \textit{symbolist}, n'impose pas d'architecture logicielle. Aussi, les applications développées avec ce framework définissent chacune leurs propres structures.
L'architecture logicielle de \textit{symbolist} n'est pas clairement définie au début de ce stage, néanmoins trois parties peuvent être distinguées dans le code source. 

Une première partie est centrée autour de la classe \textit{SymbolistHandler}. Cette classe permet l'accès depuis l'extérieur à l'application \textit{symbolist}. A savoir, l'API \textit{C} permettant à \textit{Max} et \textit{OpenMusic} de communiquer avec \textit{symbolist} appelle les méthodes de la classe \textit{SymbolistHandler} pour pouvoir interagir avec le reste de l'application.

Une deuxième partie correspond aux composants graphiques construisant l'interface de \textit{symbolist}. La fenêtre principale de l'application \textit{symbolist} est représentée par la classe \textit{SymbolistMainComponent}. Cette partie comprend également toute la hiérarchie des composants graphiques représentant les symboles de la partition.

La troisième partie de l'application regroupe les classes du \og cœur métier \fg, représentant la partition et la palette du compositeur. 
S'y trouvent les classes implémentant la structure des bundles OSC, où plus précisément des bundles \textit{odot}, qui définissent le format de données sous-jacent des symboles de la partition.
Pour décrire les symboles de la partition, l'application \textit{symbolist} possède une unique classe \textit{Symbol} (qui hérite de la classe \textit{OdotBundle}). De fait, chaque symbole de la partition, qu'il soit un rond, un carré, une note de musique, ou du texte, est représentée par une instance de la classe \textit{Symbol}. Les spécificités d'un symbole ne sont pas explicitées par sa classe d'appartenance; 
à savoir, le modèle ne prévoit pas de classes \textit{Circle}, \textit{Square} ou \textit{MusicNote}.
En effet, la philosophie de \textit{symbolist} est de conserver l'information des symboles dans les bundles OSC sous-jacents afin de pouvoir distribuer plus facilement les données de la partition.
Les classes du cœur métier incluent également toute la logique d'ordonnancement temporel des symboles associés à un \textit{staff}.
Le diagramme de classes détaillée du cœur métier de l'application \textit{symbolist} est présenté en annexe \ref{sec:symbolistModelClassDiagram}, page~\pageref{sec:symbolistModelClassDiagram}.

\paragraph{Problématiques architecturales} Plusieurs aspects posent un problème dans l'architecture de l'application \textit{symbolist} telle qu'elle se trouve au début du stage.

Premièrement, de la description des composantes de l'application \textit{symbolist} se dégage aisément une architecture de type \textit{modèle-vue-contrôleur}. Cependant, les interactions entre composantes et le rôle de chacune d'elles ne sont pas biens définis. Par exemple, la création de symboles dans \textit{symbolist}, de fait l'écriture dans le \textit{modèle}, est trop souvent réalisée par les composants graphiques du système, ce qui est contraire à l'architecture \textit{MVC}. 

Deuxièmement, même si le modèle de l'application \textit{symbolist} place les symboles de la partition à un même niveau dans la hiérarchie des classes (à savoir, il n'existe qu'une seule classe \textit{Symbol}), les composants graphiques sont eux le reflet de la structure de la partition.
C'est à dire, chaque type de composant graphique est représenté par une classe: par exemple, les cercles sont représentés par la classe \textit{CircleComponent}, les rectangles par la classe \textit{RectangleComponent}…
Aussi, comme les symboles de la partition peuvent être groupés afin de créer de nouveaux symboles, le design pattern \textit{Composite} est approprié à la hiérarchie des composants graphiques. 
Or, au début de ce stage, le design pattern \textit{Composite} n'est pas correctement appliqué. 

Troisièmement, la classe \textit{SymbolistHandler}, qui est directement liée à l'API \textit{C} rendant \textit{symbolist} utilisable par d'autres programmes, s'occupe de toutes les interactions avec les vues de l'interface: la palette, la partition, l'inspecteur… Aussi, la classe \textit{SymbolistHandler} est trop encombrée, et les interactions avec des vues spécifiques mériteraient d'être gérées par des contrôleurs spécifiques.   

  

%%%%% CHAPTER "DEVELOPPEMENT DE SYMBOLIST" %%%%%
\chapter{Développement de symbolist}
\label{chap:devSymbolist}
\todo[inline]{introduction pour le chapitre}

\section{Restructuration de l'application symbolist}
\label{sec:restructurationSymbolist}

	
%%%%% CONCLUSION %%%%%
\clearpage
\chapter{Conclusion}
Comme peut en témoigner l'Histoire, la notation musicale n'est pas un processus monolithique. L'écriture de la musique est influencée par le contexte technologique propre à chaque époque. Aussi, la musique contemporaine, qui s'installe à partir des années 50, voit ses pratiques profondément impactées par l'usage de l'électronique et l'informatique. Aujourd'hui encore, l'apparition de nouvelles technologies (machine learning, réalité virtuelle, IoT\footnote{Internet of Things}…) pousse les compositeurs à réinventer leur relation avec la musique et la manière de la noter. Au-delà de la musique contemporaine, la composition multimédia, tirant profit de la profusion des supports technologiques de notre époque (vidéo, audio, capteurs, actionneurs…), cherche encore un système de notation qui lui serait adéquat. Au moment de la rédaction de cette étude, les technologies pour la création musicale sont trop récentes et trop nombreuses pour pouvoir proposer une pratique notationnelle unique. De fait, chaque compositeur a ses propres besoins en termes d'écriture, ce qui devrait pousser les outils informatiques à proposer plus de fonctionnalités permettant l'invention d'une notation par l'utilisateur.

Or, les logiciels actuels de notation musicale ne répondent qu'en partie à la prérogative d'extensibilité ou de renouveau des pratiques d'écriture de la musique, dans le but de s'accorder avec la création contemporaine. Une première catégorie des logiciels existants intègre bien la notation traditionnelle mais ne fournit que peu de moyens pour l'étendre et l'adapter à l'écriture d'œuvres nouvelles. Une deuxième catégorie de logiciels approche la transcription musicale sous l'angle des pièces électroacoustiques et multimédias, en incorporant aux partitions une dimension interactive. Cependant, cette seconde catégorie délaisse quelque peu l'expressivité symbolique au profit de la description temporelle des processus régissant les œuvres.

Dans ce contexte, le développement du logiciel \textit{symbolist} a été initié afin d'adresser le problème du manque d'outils permettant de transcrire symboliquement les œuvres contemporaines \cite{gottfried2018}.
\textit{symbolist} est un éditeur graphique libre où le compositeur peut créer à volonté tous types de symboles et les enregistrer dans une palette. Ce logiciel est destiné à être encapsulé dans les environnements \textit{OpenMusic} et \textit{Max}, et utilise le protocole OSC pour s'interfacer avec l'extérieur.
La poursuite du développement de \textit{symbolist} constitue le cadre du présent stage. Aussi, la génèse et les caractéristiques du logiciel seront explicitées dans le rapport final d'activités.
Dans la continuité de la période d'étude bibliographique, un recueil du besoin sera effectué auprès des compositeurs de musique contemporaine (de l'IRCAM et d'ailleurs), afin de déterminer précisément les attentes des utilisateurs potentiels de \textit{symbolist}.
De plus, une démarche d'analyse et de rétro-ingénierie sera menée sur le logiciel, étant donné son état avancé de développement.
Enfin, une étape d'organisation et de planification des méthodes de travail sera préalable à l'implémentation de nouvelles fonctionnalités.   
 

\stepcounter{page}
\pagenumbering{Roman}

%%%%% BIBLIOGRAPHIE %%%%%
\printbibliography
\addcontentsline{toc}{chapter}{Bibliographie}

%%%%% TABLE DES FIGURES %%%%%
\listoffigures
\addcontentsline{toc}{chapter}{Table des figures}

%%%%% GLOSSAIRE %%%%%
\printglossary[title={Glossaire}, toctitle={Glossaire}]

%%%%% ANNEXES %%%%%
\lhead{} % remove the left part of the header (section name)
\rhead{\textit{ANNEXES}} % set right header
\appendix
\romanchapter{Annexes}
% Pour faire une référence d'une annexe
% (Annexe \ref{sec:nomsection} page~\pageref{sec:nomsection})
\section{First appendix}
\label{sec:firstAppendix}
\begin{figure}[H]
	\centering
	% \includegraphics[keepaspectratio=true, width=0.7\textwidth]{Annexes/i/refletsDeLOmbreFantaisie.jpg}
	\caption{First appendix caption}
	\label{fig:firstAppendix}	
	\medskip
	\small
	\it
	Source quotation and descriptive text.
\end{figure}
\clearpage

% Appendix declaration with a 90° rotated figure
\rotatebox{90}{
	\begin{minipage}{0.95\textheight}
		\section{Modèle de l'application symbolist}
		\label{sec:symbolistModelClassDiagram}
		\includegraphics[keepaspectratio=true, width=\textwidth]{Annexes/i/symbolistModelClassDiagram.png}
		\captionof{figure}{Diagramme de classes pour le modèle de l'application symbolist}
		\label{fig:symbolistModelClassDiagram}	
		\medskip
		\small
		\it
		En \textcolor{red}{rouge}, la classe \emph{OdotBundle}, qui encapsule la structure d'un bundle \emph{OSC}, et la classe \emph{Symbol} dont les instances représentent les symboles de la partition. Chaque symbole de la partition possède une structure de bundle OSC, d'où la relation d'héritage entre la classe \emph{OdotBundle} et \emph{Symbol}.
		En \textcolor{blue}{bleu}, la classe \emph{Palette}, regroupant les symboles pouvant être dessinés sur la partition, qu'ils aient été définis par l'utilisateur ou existaient par défaut dans l'application.
		En \textcolor{green}{vert}, les classes \emph{Score} et \emph{SortedStaves} décrivant les symboles présent dans la partition.
		En \textcolor{yellow}{jaune}, les classes \emph{SymbolTimePoint} et \emph{TimePointArray} définissant la logique d'ordonnancement temporel des symboles associés à un \emph{staff}.  
	\end{minipage}
}
\clearpage

\rotatebox{90}{
	\begin{minipage}{0.85\textheight}
		\section{Architecture finale de l'application symbolist}
		\label{sec:symbolistFinalStructure}
		\includegraphics[keepaspectratio=true, width=\textwidth]{Annexes/i/symbolistFinalStructure.png}
		\captionof{figure}{Diagramme de classes présentant l'architecture de l'application symbolist après restructuration}
		\label{fig:symbolistFinalStructure}	
		\medskip
		\small
		\it
		En \textcolor{red}{rouge}, .
		En \textcolor{blue}{bleu}, .
		En \textcolor{green}{vert}, .  
	\end{minipage}
}
\clearpage


\end{document}
