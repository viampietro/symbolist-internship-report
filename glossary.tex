\newglossaryentry{neume}
{
        name = neume,
        description = {Symbole placé au-dessus des mots d'une phrase, donnant des indications mélodiques approximatives sur leur prononciation, qu'elle soit parlée ou chantée}        
}
\newglossaryentry{portee}
{
	name = {port{é}e},
	description = {Système de lignes horizontales, parallèles et équidistantes, sur lesquelles ou entre lesquelles sont placées les notes. (Ces lignes déterminent la hauteur des notes dans l'échelle des sons. On peut leur adjoindre des lignes supplémentaires et fragmentaires au-dessus ou en dessous pour étendre l'ambitus de la portée.) - Larousse - \url{http://www.larousse.fr}}	
}
\newglossaryentry{polyphonie}
{
	name = {polyphonie},
	description = {Combinaison de plusieurs voix ou parties mélodiques, dans une composition musicale - Wikipédia}	
}
\newglossaryentry{tonalite}
{
	name = {tonalit{é}},
	description = {Une tonalité se définit comme une gamme de sept notes, désignée par sa tonique (appartenant à l'échelle diatonique) et son mode (majeur ou mineur) - Wikipédia}
}

\newglossaryentry{clef}
{
	name = {clef},
	description = {Signe placé au commencement de la portée et qui détermine le nom des notes, ainsi que leur hauteur précise dans l'échelle musicale. (Il y a trois figures de clés : clés de fa, de sol et d'ut.) - Larousse - \url{http://www.larousse.fr}}
}

\newglossaryentry{sigrythmique}
{
	name = {signature rythmique},
	description = {En solfège, il s’agit du chiffrage que l’on trouve généralement au début de la portée à droite de la clé. La signature rythmique se compose généralement de 2 chiffres :
le chiffre du haut indique combien de temps (unité de durée en musique) il y a dans chaque mesure; 
le chiffre du bas indique, quant à lui, quelle est la note qui vaut 1 temps (1 = ronde, 2 = blanche, 4 = noire, 8 = croche, 16 = double croche, …) - acadezik - \url{https://www.acadezik.com}}
}

\newglossaryentry{gammechromatique}
{
	name = {gamme chromatique},
	description = {Gamme formée par une succession de demi-tons - Larousse - \url{http://www.larousse.fr}}
}

\newglossaryentry{liturgique}
{
	name = {liturgique},
	description = {Relatif ou conforme à la liturgie. Le mot liturgie désigne l'ensemble des rites, cérémonies et prières dédiés au culte d'une divinité religieuse, tels qu'ils sont définis selon les règles éventuellement codifiées dans les textes sacrés ou la tradition - Wikipédia}
}