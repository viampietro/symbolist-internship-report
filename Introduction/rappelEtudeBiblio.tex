La présente section se veut être une synthèse des problématiques soulevées dans l'étude bibliographique préalable à ce mémoire.

Comme vu précédemment, la musique contemporaine, stimulée par la profusion technologique du XXème et du XXIème siècle, voit apparaître un ensemble de nouvelles pratiques de composition et d'interprétation \cite{bosseur2005}.
Les pratiques suivantes peuvent être identifiées comme nouvellement amenées ou fortement approfondies par la musique contemporaine:

\begin{enumerate}[label={(\arabic*)}]
	\item \textbf{Utilisation de l'électronique} Les techniques de captation, de transformation et de diffusion du son via des médiums électroniques n'ont eu de cesse de joncher les productions de la musique nouvelle.
	Les problématiques de spatialisation des sons, c'est à dire de leur placement dans un espace de diffusion à deux ou trois dimensions, est un bon exemple de nouvelles pratiques liées à l'électronique \cite{harley1993}.
	
	\item \textbf{Utilisation de l'informatique} La musique contemporaine en fait un usage éten-\\du, dans sa dimension logicielle et programmationnelle, c'est à dire, aussi bien via l'utilisation de programmes \textit{wysiwyg} qu'à travers les langages de programmation dédiés. L'informatique intervient dans le processus de composition assistée par ordinateur (CAO) pour le calcul et l'invention de structures musicales \cite{agon1998}, ou, dans une logique de création d'instruments virtuels ou de productions sonores directes \cite{puckette1991, mccartney1996}. Un dernier point qui donne un intérêt à l'informatique dans la composition et l'exécution de musique nouvelle réside dans la capacité de communication et de synchronisation permises entre l'homme et la machine dans le cas de musique \textit{mixte}~\cite{cont2008}.
	
	\item \textbf{Nouvelle approche du geste} Les œuvres de musique contemporaine et les performances multimédias (en particulier les performance multimédias) font un usage autre des instruments de musique. La pièce \textit{Fluoresce} (2012) de Rama Gottfried peut-être prise en exemple; durant son éxecution, le joueur de violoncelle va frotter les cordes avec un fil ou encore va les soulever avec plus ou moins de force tout en les frottant. Cette réinvention de la manière de jouer des instruments vient de la recherche constante de nouvelles matières sonores, et même l'invention de nouveaux instruments, par les compositeurs contemporains, et de la considération du geste du musicien comme étant fondamentalement signifiant dans l'exécution d'une pièce \cite{zanpronha2005}.
	
	\item \textbf{Nouvelle approche du temps} En plus de faire évoluer les parties instrumentales à des vitesses différentes (polyrythmie), la musique contemporaine fait la distinction entre deux types de temporalité: le temps quantifié (ou réel) et le temps proportionnel \cite{honing1993}. Une pièce musicale en temps quantifié verra la durée de ses éléments fixée préalablement. Par exemple, le temps mesuré est une type de temps quantifié, où la durée de chaque évènement musical est défini proportionnellement à un nombre donné de battements par minute. Une pièce en temps proportionnel fixe les relations de durée de ses éléments les uns par rapport aux autres, mais omet la définition absolue des durées. 
\end{enumerate}

Pour autant, les compositeurs contemporains possèdent toujours un lieu privilégié pour le travail, la transcription et la transmission de leur musique: la partition \cite{bresson2008}.
Aujourd'hui, la pluralité des formes de représentation de la musique posent la question de l'identité de la partition. En effet, la \textit{Common Western Music Notation} (CWMN), pratique notationnelle enseigné dans les conservatoires, n'est plus l'unique système permettant de noter la musique. Cependant, les nouveaux systèmes de visualisation de la musique acquiert-il pour autant le statut de systèmes d'écriture? 
Un système d'écriture pour la musique peut être identifié comme tel à la condition de posséder les critères suivants \cite{veitl2007}:

\begin{enumerate}[label={(\arabic*)}]
	\item \textbf{La matérialité}, ou le fait d'être conservé sur un support physique: papier, mémoire électronique…	
	\item \textbf{La visibilité}, ou le fait d'exposer à la vue son contenu (les bits stockés dans un disque dur sans visualisation ne constitue par un système d'écriture).
	\item \textbf{La lisibilité}, ou sa capacité de compréhension par un être humain, ou une machine.
	\item \textbf{Le caractère performatif}, ou la capacité de transformation de l'information en son, par une machine ou une être humain.
	\item \textbf{Le caractère systémique}, ou la proposition d'un système organisé de \og signes discrets \fg.
\end{enumerate}

En répondant à ces critères, les partitions graphiques des années cinquante/soixante de John Cage, Earle Brown, David Tudor et bien d'autres renouvellent déjà le système d'écriture en s'intéressant à l'\og image-son \fg, une manière de représenter la musique par des graphismes \textit{pseudo-arbitraires} \cite{saladin2004}, proposant un système de symboles propres à chaque compositeur.
De plus, avec l'arrivée du format MIDI \cite{midi1996}, la visualisation des morceaux de musique sous forme de \textit{piano-roll} se démocratise, apportant également un nouveau mode de notation. 
Aussi, le code informatique, décrivant de manière procédurale, en CAO, comment créer des structures musicales \cite{agon1998}, ou comment générer la musique avec un objectif d'exécution temps-réel \cite{puckette1991, mccartney1996, cont2008}, participe aussi d'une vision renouvelée de ce qu'est la partition.
Enfin, les représentations en formes d'ondes et en spectrogrammes, proche d'une description morphologique du son, ainsi que les courbes d'automation pilotant les effets appliqués, pourraient également constituées une partition alternative. Cependant, leur manque de lisibilité par l'être humain les relègue à un système de représentation plus qu'à un système d'écriture \cite{veitl2007}.

De paire avec cette pluralité des modes de fixation de la musique vient la pluralité des outils de notation. 
Un des supports pour la notation le plus utilisé par les compositeurs de musique contemporaine, du moins dans leur travail préliminaire, reste le papier. La feuille blanche n'impose aucune limitation de format, si ce n'est par la taille du support, et le compositeur n'est alors bridé que par sa propre expressivité.
Ensuite, les logiciels pour la notation de la musique se déclinent en deux catégories du point de vue de l'interface homme-machine: logiciels \textit{wysiwyg} et logiciels à compilateur (ou alors un peu des deux pour les logiciels \textit{wysiwyg} proposant des fonctionnalités de \textit{scripting}) \cite{fober2015}.
Au delà de cette classification par l'IHM, qui est du moins la plus franche, les logiciels de notation suivent trois tendances:
\begin{enumerate}[label=(\arabic*)]
	\item \textbf{Programmes orientés CWMN} Cette catégorie regroupe les applications dont l'interface et les fonctionnalités sont construites sur le système d'écriture musical-traditionnel-occidental. L'environnement graphique de telles applications est centré autour d'une portée (au sens classique de la construction à cinq lignes). En termes de possibilités d'écriture, l'utilisateur peut aisément transcrire des pièces musicales allant de l'époque du chant grégorien jusqu'aux œuvres du début du XXème siècle. De plus, un rendu audio de la partition est souvent possible, par le biais d'une conversion au format MIDI, ce qui permet au compositeur d'obtenir un aperçu de sa pièce. Certaines applications comme \textit{Lilypond} \cite{lilypond2018} et \textit{Finale} \cite{finale2018} proposent d'étendre la grammaire des symboles utilisables dans le processus d'écriture. Cependant, deux problèmes émergent lors de la définition de nouveaux symboles par l'utilisateur: Premièrement, ces nouveaux symboles ne sont pas alignés automatiquement avec les autres éléments de la partition, et il n'est pas possible de fixer des points d'ancrage sur ces nouvelles figures pour arriver à un tel comportement. Deuxièmement, il n'existe pas de rendu audio pour les symboles de l'utilisateur, et il n'est pas possible de lier un nouveau symbole à un évènement MIDI ou un fichier audio pour donner la possibilité de son aperçu sonore. En clair, aucun des logiciels orientés CWMN ne propose une manière d'attacher une sémantique aux symboles créés.
Pour conclure, les programmes orientés CWMN sont nombreux (\textit{Sibelius} \cite{sibelius2018}, \textit{MuseScore} \cite{musescore2018}, \textit{Dorico} \cite{dorico2018}, \textit{Noteability Pro} \cite{noteAbility2018} et bien d'autres sont encore à citer) et offrent un large panel de fonctionnalités pour noter exhaustivement la musique occidentale du XIème jusqu'au début du XXème siècle.
Néanmoins, ces programmes se cantonnent à ce seul paradigme et ne permettent pas ou très difficilement de s'en écarter. Du point de vue de la musique contemporaine et des nouvelles pratiques (citées ci-dessus) qu'elle apporte, les applications CWMN sont d'autant plus limitées. En effet, étant donné l'absence d'éléments standardisés pour la notation de la spatialisation, des gestes du musicien, des scripts informatiques, etc., les compositeurs de musique contemporaine ayant recours aux logiciels orientés CWMN sont obligés d'inventer de nouveaux symboles, or les limitations induites par cette méthode ont été exposées précédemment.
  
	\item \textbf{Programmes orientés notation de l'électroacoustique} 	
	\item \textbf{Programmes à la croisée des chemins}
\end{enumerate} 

