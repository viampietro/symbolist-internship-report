La présente section synthétise les problématiques soulevées dans l'étude bibliographique préalable à ce mémoire.

Comme vu précédemment, la musique contemporaine, stimulée par la profusion technologique du XXème et du XXIème siècle, voit apparaître un ensemble de nouvelles pratiques de composition et d'interprétation \cite{bosseur2005}.
Les pratiques suivantes peuvent être identifiées comme nouvellement amenées ou fortement approfondies par la musique contemporaine:

\begin{enumerate}[label={(\arabic*)}]
	\item \textbf{Utilisation de l'électronique.} Les techniques de captation, de transformation et de diffusion du son via des médiums électroniques n'ont eu de cesse de joncher les productions de la musique nouvelle\footnote{Synonyme de musique contemporaine savante}.
	Les problématiques de spatialisation des sons, c'est à dire de leur placement dans un espace de diffusion\footnote{Un espace de diffusion est le résultat sonore d'une installation des haut-parleurs. Par exemple, un espace de diffusion stéréophonique est créé par l'utilisation de deux haut-parleurs placés à gauche et à droite de l'auditeur.}, est un bon exemple de nouvelles pratiques liées à l'électronique \cite{harley1993}.
	
	\item \textbf{Utilisation de l'informatique.} La musique contemporaine en fait un usage éten-\\du, dans sa dimension logicielle et programmationnelle, c'est à dire, aussi bien via l'utilisation de programmes \textit{wysiwyg}\footnote{What You See Is What You Get} qu'à travers les langages de programmation dédiés. L'informatique intervient dans le processus de composition assistée par ordinateur (CAO) pour le calcul et l'invention de structures musicales \cite{agon1998}, ou dans une logique de création d'instruments virtuels ou de production sonore \cite{puckette1991, mccartney1996}. Un dernier point qui donne un intérêt à l'informatique dans la composition et l'exécution de musique nouvelle réside dans la capacité de communication et de synchronisation permises entre l'homme et la machine dans le cas de musique \textit{mixte}~\cite{cont2008}.
	
	\item \textbf{Nouvelle approche du geste.} Les œuvres de musique contemporaine et les performances multimédias font un usage non-conventionnel des instruments de musique. A savoir, les modes de production sonore naturels des instruments ne sont plus privilégiés. La pièce \textit{Fluoresce} (2012) de Rama Gottfried peut-être prise en exemple; durant son éxecution, le joueur de violoncelle va frotter les cordes avec un fil ou encore va les soulever avec plus ou moins de force tout en les frottant. Cette réinvention de la manière de jouer des instruments vient de la recherche constante de nouvelles matières sonores, poussant même les compositeurs contemporains à inventer de nouveaux instruments. Le geste du musicien est alors considéré comme fondamentalement signifiant dans l'exécution d'une pièce, du fait de son caractère exploratoire \cite{zanpronha2005}.
	
	\item \textbf{Nouvelles approches du temps.} En plus de faire évoluer les parties instrumentales à des vitesses différentes (polyrythmie), la musique contemporaine fait la distinction entre plusieurs types de temporalité, par exemple: le temps quantifié (ou réel) et le temps proportionnel \cite{honing1993}. Une pièce musicale en temps quantifié verra la durée de ses éléments fixée préalablement. Par exemple, le temps mesuré est une type de temps quantifié, où la durée de chaque évènement musical est défini proportionnellement à un nombre donné de battements par minute. Une pièce en temps proportionnel fixe les relations de durée de ses éléments les uns par rapport aux autres, mais omet la définition absolue des durées. 
\end{enumerate}

Pour autant, les compositeurs contemporains possèdent toujours un lieu privilégié pour le travail, la transcription et la transmission de leur musique: la partition \cite{bresson2008}.
Aujourd'hui, la pluralité des formes de représentation de la musique posent la question de l'identité de cette partition. En effet, la \textit{Common Western Music Notation} (CWMN), pratique notationnelle enseignée dans les conservatoires, n'est plus l'unique système permettant de noter la musique. Cependant, les nouveaux systèmes de visualisation de la musique acquièrent-ils pour autant le statut de systèmes d'écriture? 
Un système d'écriture pour la musique peut être identifié comme tel à la condition de posséder les critères suivants \cite{veitl2007}:

\begin{enumerate}[label={(\arabic*)}]
	\item \textbf{La matérialité}, ou le fait d'être conservé sur un support physique: papier, mémoire électronique…	
	\item \textbf{La visibilité}, ou le fait d'exposer à la vue son contenu (les bits stockés dans un disque dur sans visualisation ne constitue par un système d'écriture).
	\item \textbf{La lisibilité}, ou sa capacité de compréhension par un être humain, ou une machine.
	\item \textbf{Le caractère performatif}, ou la capacité de transformation de l'information en son, par une machine ou une être humain.
	\item \textbf{Le caractère systémique}, ou la proposition d'un système organisé de \og signes discrets \fg, ou \og symboles \fg.
\end{enumerate}

En répondant à ces critères, les partitions graphiques des années cinquante/soixante de John Cage, Earle Brown, David Tudor et bien d'autres renouvellent déjà le système d'écriture en s'intéressant à l'\og image-son \fg, une manière de représenter la musique par des graphismes \textit{pseudo-arbitraires} \cite{saladin2004}, proposant un système de symboles propres à chaque compositeur.
De plus, avec l'arrivée du format MIDI\todo{gls} \cite{midi1996}, la visualisation des morceaux de musique sous forme de \textit{piano-roll}\todo{gls}\footnote{Le piano roll, dans son acception moderne, est un mode de visualisation et d’édition de notes MIDI, inspiré des rouleaux de papier perforé autrefois utilisés dans les pianos mécaniques. Les touches du piano sont représentés sur l’axe vertical, et la ligne du temps sur l’axe horizontal. Une note MIDI est représenté dans la grille par un rectangle de longueur variable.} se démocratise, apportant également un nouveau mode de notation. 
Aussi, le code informatique, décrivant de manière procédurale comment créer des structures musicales \cite{agon1998}, ou comment générer la musique avec un objectif d'exécution temps-réel \cite{puckette1991, mccartney1996, cont2008}, participe d'une vision renouvelée de la notation d'une œuvre.
Enfin, les représentations en formes d'ondes et en spectrogrammes, proche d'une description morphologique du son, ainsi que les courbes d'automation\todo{gls} pilotant les effets appliqués, pourraient également constituer une partition alternative. Cependant, leur manque de lisibilité par l'être humain les relègue à un système de représentation plus qu'à un système d'écriture \cite{veitl2007}.

De paire avec cette pluralité des modes de fixation de la musique vient la pluralité des outils de notation. 
Un des supports pour la notation le plus utilisé par les compositeurs de musique contemporaine, du moins dans leur travail préliminaire, reste le papier. La feuille blanche n'impose aucune limitation de format, si ce n'est par la taille du support, et le compositeur n'est alors bridé que par sa propre expressivité.
Ensuite, les logiciels pour la notation de la musique se déclinent en deux catégories du point de vue de l'interface homme-machine: logiciels \textit{wysiwyg} et logiciels à compilateur (ou alors un peu des deux pour les logiciels \textit{wysiwyg} proposant des fonctionnalités de \textit{scripting}) \cite{fober2015}.
Au delà de cette classification par l'IHM, qui est du moins la plus franche, les logiciels de notation suivent trois tendances:
\begin{enumerate}[label=(\arabic*)]
	\item \textbf{Programmes orientés CWMN.} Cette catégorie regroupe les applications dont l'interface et les fonctionnalités sont construites sur le système d'écriture musical-traditionnel-occidental. L'environnement graphique de telles applications est centré autour d'une portée classique. En termes de possibilités d'écriture, l'utilisateur peut aisément transcrire des pièces musicales allant de l'époque du chant grégorien jusqu'aux œuvres du début du XXème siècle. De plus, un rendu audio de la partition est souvent possible, par le biais d'une conversion au format MIDI, ce qui permet au compositeur d'obtenir un aperçu de sa pièce. Certaines applications comme \textit{Lilypond} \cite{lilypond2018} et \textit{Finale} \cite{finale2018} proposent d'étendre la grammaire des symboles utilisables dans le processus d'écriture. 
	
	Cependant, deux problèmes émergent lors de la définition de nouveaux symboles par l'utilisateur: Premièrement, ces nouveaux symboles ne sont pas alignés automatiquement avec les autres éléments de la partition, et il n'est pas possible de fixer des points d'ancrage sur ces nouvelles figures pour arriver à un tel comportement. Deuxièmement, il n'existe pas de rendu audio pour les symboles de l'utilisateur, et il n'est pas possible de lier un nouveau symbole à un évènement MIDI ou un fichier audio pour donner la possibilité de son aperçu sonore. En clair, aucun des logiciels orientés CWMN ne propose une manière d'attacher une sémantique aux symboles créés.
Pour conclure, les programmes orientés CWMN sont nombreux (\textit{Sibelius} \cite{sibelius2018}, \textit{MuseScore} \cite{musescore2018}, \textit{Dorico} \cite{dorico2018}, \textit{Noteability Pro} \cite{noteAbility2018} et bien d'autres sont encore à citer) et offrent un large panel de fonctionnalités pour noter exhaustivement la musique occidentale de manière \og classique \fg.
Néanmoins, ces programmes se cantonnent à ce seul paradigme et ne permettent pas ou très difficilement de s'en écarter. Du point de vue de la musique contemporaine et de ses nouvelles pratiques (citées ci-dessus), les applications CWMN sont d'autant plus limitées. 
  
	\item \textbf{Programmes orientés \og notation de l'électroacoustique \fg.} Une partie des logiciels dédiés à l'écriture de la musique électroacoustique servent la définition de scénarios d'exécution ou de séquences d'évènements pour les processus électroniques et informatiques \cite{arias2017, coduys2003, cont2008}. Ces logiciels s'intéresse à la notation de l'évolution des paramètres de contrôle de l'environnement électronique/informatique d'une œuvre. Les scénarios d'exécution définis doivent pouvoir être communiqués aux processus électroniques/informatiques responsables de la production du son: une décomposition est effectuée, comme pour les humains, entre la partition (le scénario) et l'interprète \cite{pope1986}.
	L'interprète d'une pièce peut être un programme informatique dont non seulement le comportement, mais également la structure, doivent être définis. Il n'est donc pas dénué de sens de se demander si les langages informatiques dédiés à la création d'instruments de synthèse sonore (des potentiels interprètes), comme \textit{Max} \cite{puckette1991} ou \textit{SuperCollider} \cite{mccartney1996}, ne font pas également partie du système de notation des pièces électroacoustiques. A certains d'arguer que les langages de synthèse sonore font partie du paradigme du temps réel, et qu'ils sont dès lors décorrélés du temps compositionnel, écartant leur considération en tant que systèmes de notation \cite{risset1999}. D'autres pensent que ces langages, visuels ou textuels, sont bien des systèmes d'écriture de la musique, mais qu'au même titre que les formes d'ondes ou les spectrogrammes leur manque d'expressivité en font un support inerte pour l'être humain \cite{gottfried2017}.
	Le sujet est toujours en débat dans les groupes de travail s'intéressant à la notation de la musique contemporaine.
	Pour en revenir aux programmes permettant l'écriture du \og scénario \fg d'une pièce électroacoustique, une de leur caractéristique est l'utilisation du temps non-mesuré. L'unité temporelle préférée, pour associer une durée aux éléments du scénario, est donc la seconde plutôt que les valeurs rythmiques de la CWMN.
	
	Le principal reproche qui peut être fait à ce type de logiciels est leur absence d'intégration, pour la plupart, des symboles de la CWMN dans leur système d'écriture. Aussi, les logiciels comme \textit{Iannix} \cite{coduys2003} ou \textit{Ossia} (ancien \textit{i-score} \cite{assayag2008}) ne permettent pas d'étendre la grammaire des symboles utilisables dans le processus d'écriture de la pièce. Enfin, pour reprendre une observation faite plus haut, les partitions produites par ces logiciels sont certes très bien exécutées par des interprètes informatiques, mais posent la question de leur lisibilité par un être humain. Dans le cas de musique mixte, de telles partitions peuvent-elles constituer un appui pour  l'interprète humain afin de se synchroniser avec la machine? Visiblement, l'expressivité d'un système notationnel est toujours mise au défi lorsque plusieurs profils d'interprètes sont impliqués dans l'exécution d'une pièce. De fait, la question \og pour qui écrit-on? \fg est fondamentale lors de la conception d'une partition, et plusieurs profils d'interprètes tendraient à considérer la création de plusieurs types de partitions \cite{pope1986}.
	
	\item \textbf{Programmes mixtes.} Il existe un troisième type de programmes, qui ne se centre pas sur un des deux paradigmes notationnels présentés ci-avant, mais les fait cohabiter au sein d'un même environnement.
	Or, à ce jour, le seul exemple connu d'un tel programme est l'environnement \textit{INScore} \cite{fober2012}. Celui-ci permet d'afficher librement, dans un canevas \og feuille blanche \fg, tous types de graphismes: des signaux sonores, des formes géométriques, des portées et des symboles de la CWMN…
	Plusieurs caractéristiques propres à \textit{INScore}, qui peuvent sûrement être étendues à tout programme de notation mixte, répondent à la nécessité de bonne cohabitation de plusieurs paradigmes notationnels.
	Par exemple, les mécanismes pour l'alignement des éléments graphiques évoluant dans des temporalités différentes, mais aussi l'interopérabilité, le rendu audio, ou l'intervention de l'utilisateur sur le contenu de la partition lors de son exécution (\textit{modifications temps-réel}) comptent parmi les possibilités du logiciel.
\end{enumerate}

Pour résumé, le manque d'outils transversaux, qui permettraient d'écrire les nouvelles pratiques musicales comme les anciennes, est une problématique phare pour le développement de nouveaux logiciels musicaux.
En effet, un besoin existe pour une multiplicité des formes de visualisation d'une partition, et ce selon plusieurs critères:
\begin{itemize}[label=--]
	
	\item A qui s'adresse la partition? Est-elle à destination d'un humain, d'une machine, ou les deux?
	
	\item Quel est le but de cette partition? A-t-elle pour finalité d'être lue et exécutée? Ou bien est-elle un support pour le travail compositionnel sans prétention de diffusion du résultat?
	
	\item Constitue-t-elle un support échangeable? Doit-elle subir une transformation avant d'être comprise par son destinataire? Cela rejoint la question de l'interopérabilité d'une partition, notamment dans le cas de son interprétation par une machine.
	
\end{itemize}
