La présente section se veut être une synthèse des problématiques soulevées dans l'étude bibliographique préalable à ce mémoire.

Comme vu précédemment, la musique contemporaine, stimulée par la profusion technologique du XXème et du XXIème siècle, voit apparaître un ensemble de nouvelles pratiques de composition et d'interprétation, ces deux dernières étant toujours fortement liées \cite{bosseur2005}.
Les pratiques suivantes peuvent être identifiées comme nouvellement amenées ou fortement approfondies par la musique contemporaine:

\begin{itemize}[label=--]
	\item \textbf{Utilisation de l'électronique} les techniques de captation, de transformation et de diffusion du son via des médiums électroniques n'ont eu de cesse de joncher les productions de musique nouvelle.
	Les problématiques de spatialisation des sons, c'est à dire de leur placement dans un espace de diffusion, est un bon exemple de nouvelles pratiques liées à l'électronique \cite{harley1993}.
	\item \textbf{Utilisation de l'informatique} dans sa dimension logicielle et programmationnelle, c'est à dire, aussi bien via l'utilisation de programmes \textit{wysiwyg} qu'à travers les langages de programmation dédiés. L'informatique intervient dans le processus de composition assistée par ordinateur pour le calcul et l'invention de structures musicales \cite{agon1998}, ou, dans une logique de création d'instruments virtuels ou de productions sonores directes \cite{puckette1991, mccartney1996}. Un dernier point qui donne un intérêt à l'informatique dans la composition et l'exécution de musique nouvelle réside dans la capacité de communication et de synchronisation permises entre l'homme et la machine dans le cas de musique \textit{mixte} \cite{cont2008}. 
\end{itemize}