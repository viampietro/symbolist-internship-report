Dans son livre \citetitle{sharples1998}, Mike Sharples présente le \og cycle d'engagement et de réflexion qui forme la mécanique de l'écriture \fg. Ce cycle est constitué de quatre étapes: la première étape est l'étape de \textit{contemplation} qui représente le moment où les idées se forment dans l'esprit de l'écrivain; la deuxième étape est l'étape de \textit{spécification} pendant laquelle l'écrivain va sélectionner et organiser ses idées, toujours dans son esprit; la troisième étape correspond à l'écriture concrète, à la \textit{génération} du texte sur un support physique ou numérique; enfin, la dernière étape est l'étape d'interprétation où l'écrivain opère une revue de la matière écrite produite. Au cours de la troisième étape, celle de la génération du texte, l'outil d'expression écrite utilisé par l'écrivain prend toute son importance. Il doit permettre l'expression structurée de la pensée de l'auteur, et s'insérer dans le cycle de l'écriture sans contraindre la fluidité du processus de création. A cet égard, la conception d'outils pour la \og création \fg soulèvent de nombreux défis. En effet, une idée, à la durée de vie éphémère dans l'esprit humain, s'accommode et est influencée par les outils qui servent à la formuler - signifiant littéralement lui donner forme. Aussi, un \og mauvais \fg design d'outils pourrait brider la créativité des utilisateurs en retenant leur expressivité. Alors, comment créer un outil garant de la pensée du créateur?  

Dans le domaine de l'informatique musicale, l'équipe \textit{Représentations Musicales} de l'Ircam s'intéresse aux problématiques de composition assistée par ordinateur (CAO). La recherche en CAO est très appliquée; sa finalité est toujours la création d'outils assistant les compositeurs dans la formalisation de leurs idées musicales. Dans cette optique, les logiciels de notation musicale prennent une place prépondérante. Aussi, la question du design de tels outils est un réel sujet d'étude, puisqu'ils impactent directement la pensée compositionnelle \cite{fober2015}. Dans le domaine de la recherche en informatique musicale, les outils de notation ont plusieurs objectifs: premièrement, permettre aux compositeurs de noter leurs pièces telles qu'il les imagine; d'un point de vue pragmatique, l'outil doit être à la fois expressif et facile d'utilisation, selon les critères du domaine \cite{nash2015}. Deuxièmement, faire avancer la recherche, au sens où un outil informatique est aussi un lieu d'expérimentations. Dans le cas de la notation musicale, la conception d'un logiciel est l'occasion pour développer et tester de nouveaux modèles d'écriture. Également, la partition, dans son format physique ou numérique, constitue à elle seule un canevas pour la recherche musicale. Naturellement, la recherche en \og informatique musicale \fg s'intéresse à l'expression numérique de ce canevas et aux possibilités qu'elle apporte.
De fait, c'est dans cette démarche que l'outil de notation \textit{symbolist}, objet central de travail et d'étude de ce stage, a été façonné.

Ce mémoire revient, dans un premier temps, sur les problématiques notationnelles de la musique contemporaine et expose de manière synthétique l'état de l'art sur les outils de notation présenté dans l'étude bibliographique préliminaire. Ensuite, l'application \textit{symbolist}, issue d'une collaboration entre l'Ircam et la HfMT\footnote{Hochschule für Musik und Theater} d'Hambourg, est présentée. Après, la contribution technique à la mise en place d'une architecture logicielle pour \textit{symbolist} est décrite. Puis, la démarche de recherche autour des thématiques de la notation en informatique musicale est exposée à travers deux sujets: le concept de notation exécutable et le renouvellement du modèle d'écriture musicale pour l'application \textit{symbolist}. Enfin, un bilan des activités du stage et les perspectives qu'il a ouverte sont explicitées dans la conclusion.



