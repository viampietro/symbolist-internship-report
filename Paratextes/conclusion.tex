L'explosion des formes de composition et de production de la musique au XXème et XXIème siècle amène la réflexion suivante: est-ce qu'un outil de notation peut tout proposer, et combler les attentes de tous en terme d'expression des idées musicales? L'existence d'un tel outil est-elle nécessaire, enviable?
Étant donné la perméabilité de la relation entre pensée compositionnelle et outils de composition, un outil omnipotent et unique n'amènerait-il pas à la constitution d'une pensée unique, et donc à une antithèse de la créativité?

Dans le paysage des outils de notation pour la musique, \textit{symbolist} propose une approche parmi d'autre pour l'écriture de la musique. \textit{symbolist} se concentre sur l'aspect du dessin libre et sur la capacité d'échange des informations contenues dans une partition entre processus informatiques.
Durant ce stage, une architecture logicielle stable, facilitant la contribution d'autres personnes au développement du projet \textit{symbolist}\footnote{Le code source du projet est accessible sur GitHub à l'adresse \url{https://github.com/ramagottfried/symbolist}}, a d'abord été mise en place. Par la suite, dans une démarche pluridisciplinaire privilégiant l'interaction avec les compositeurs de musique contemporaine, de nouvelles fonctionnalités ont été imaginées pour l'application \textit{symbolist}. Ce faisant, la démarche de création de nouvelles fonctionnalités a été un terreau fertile pour la recherche informatique, et a nourri la réflexion autour de problématiques centrales de la notation musicale.