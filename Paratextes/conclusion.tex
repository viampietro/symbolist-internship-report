Comme peut en témoigner l'Histoire, la notation musicale n'est pas un processus monolithique. L'écriture de la musique est influencée par le contexte technologique propre à chaque époque. Aussi, la musique contemporaine, qui s'installe à partir des années 50, voit ses pratiques profondément impactées par l'usage de l'électronique et l'informatique. Aujourd'hui encore, l'apparition de nouvelles technologies (machine learning, réalité virtuelle, IoT\footnote{Internet of Things}…) pousse les compositeurs à réinventer leur relation avec la musique et la manière de la noter. Au-delà de la musique contemporaine, la composition multimédia, tirant profit de la profusion des supports technologiques de notre époque (vidéo, audio, capteurs, actionneurs…), cherche encore un système de notation qui lui serait adéquat. Au moment de la rédaction de cette étude, les technologies pour la création musicale sont trop récentes et trop nombreuses pour pouvoir proposer une pratique notationnelle unique. De fait, chaque compositeur a ses propres besoins en termes d'écriture, ce qui devrait pousser les outils informatiques à proposer plus de fonctionnalités permettant l'invention d'une notation par l'utilisateur.

Or, les logiciels actuels de notation musicale ne répondent qu'en partie à la prérogative d'extensibilité ou de renouveau des pratiques d'écriture de la musique, dans le but de s'accorder avec la création contemporaine. Une première catégorie des logiciels existants intègre bien la notation traditionnelle mais ne fournit que peu de moyens pour l'étendre et l'adapter à l'écriture d'œuvres nouvelles. Une deuxième catégorie de logiciels approche la transcription musicale sous l'angle des pièces électroacoustiques et multimédias, en incorporant aux partitions une dimension interactive. Cependant, cette seconde catégorie délaisse quelque peu l'expressivité symbolique au profit de la description temporelle des processus régissant les œuvres.

Dans ce contexte, le développement du logiciel \textit{symbolist} a été initié afin d'adresser le problème du manque d'outils permettant de transcrire symboliquement les œuvres contemporaines \cite{gottfried2018}.
\textit{symbolist} est un éditeur graphique libre où le compositeur peut créer à volonté tous types de symboles et les enregistrer dans une palette. Ce logiciel est destiné à être encapsulé dans les environnements \textit{OpenMusic} et \textit{Max}, et utilise le protocole OSC pour s'interfacer avec l'extérieur.
La poursuite du développement de \textit{symbolist} constitue le cadre du présent stage. Aussi, la génèse et les caractéristiques du logiciel seront explicitées dans le rapport final d'activités.
Dans la continuité de la période d'étude bibliographique, un recueil du besoin sera effectué auprès des compositeurs de musique contemporaine (de l'IRCAM et d'ailleurs), afin de déterminer précisément les attentes des utilisateurs potentiels de \textit{symbolist}.
De plus, une démarche d'analyse et de rétro-ingénierie sera menée sur le logiciel, étant donné son état avancé de développement.
Enfin, une étape d'organisation et de planification des méthodes de travail sera préalable à l'implémentation de nouvelles fonctionnalités.   
 