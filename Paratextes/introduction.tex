La notation musicale peut être distinguée en deux approches : une approche prescriptive et une approche descriptive \cite{battier2015}.
La notation prescriptive a pour but de décrire \og comment la musique doit sonner \fg.
Dans cette optique, la partition fait office de référence ou du moins de repère pour l'interprétation d'une pièce. 
La notation descriptive tente de retranscrire \og comment la musique a sonné \fg.
Ainsi, à des fins d'analyse, une pièce peut être caractérisée et fixée sur la \gls{portee}.
Également, la partition est l'outil privilégié du compositeur pour la communication de sa musique, et, au-delà de l'objet fini, constitue un espace de travail.
Pour donner un exemple, la préparation d'une pièce contemporaine par un ensemble musical se fait souvent en collaboration avec le compositeur. La partition constitue alors le support de la discussion et se voit même être modifiée pour les besoins de l'exécution de la pièce. Comme le dit Carmine E. Cella, chercheur et compositeur à l'IRCAM, en parlant de l'écriture musicale : \og le créateur doit s'efforcer de trouver un compromis notationnel entre sa pensée et la réalisation pratique de sa pièce \fg. L'annexe \ref{sec:refletsDeLOmbre} donne deux exemples de partitions représentant la même partie de la pièce \textit{Reflets de l'ombre} (C. E. Cella, 2013), montrant l'adaptation de la notation à des fins d'exécution.

Aussi, procurer des outils aux compositeurs pour leur permettre de noter leurs œuvres est un enjeu primordial. Bien sûr, ces outils ont besoin d'être en phase avec les pratiques de création de leur époque, pour ne pas brider la créativité des utilisateurs.
Or, la musique contemporaine savante et la composition multimédia soulèvent de nouveaux défis d'écriture, en apportant des pratiques inédites, fortement liées à l'usage massif de l'informatique et de l'électronique par les compositeurs. Par conséquent, la présente étude dresse un état de l'art des outils informatiques pour la notation musicale en se posant la question de la pertinence de ces outils pour la transcription des pièces contemporaines.

D'abord, la notation musicale sera décrite sous un axe historique, en s'attardant sur la période  contemporaine et les problématiques notationnelles qu'elle soulève. Ensuite, une vue d'ensemble des moyens informatiques pour la création musicale sera présentée, afin de donner un aperçu au lecteur de l'écosystème actuel dans lequel les compositeurs évoluent et qui influence de fait leurs pratiques. Puis, les outils informatiques existants pour une notation symbolique et traditionnelle de la musique seront détaillés, et leurs caractéristiques seront discutées sous l'angle de la problématique. Enfin, la dernière section s'intéressera aux logiciels et environnements abordant la notation musicale sous un axe contemporain, apportant une dimension interactive aux partitions.

Pour le confort de lecture et la compréhension du rapport, les mots du vocabulaire musical apparaissent en bleu et sont répertoriés dans le glossaire à la fin du document (voir \nameref{main}).


